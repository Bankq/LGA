\documentclass[10pt]{report}

\setcounter{secnumdepth}{1}

\title{\textbf{LGA Language Reference Manual - V0.1}}
\author{ Hang Qian \\ Pindan Hao \\ Yuanli Dong \\ Tian Xia \\ {\small \texttt{\{hq2124, ph2389, yd2270, tx2126\}@columbia.edu}}}
\date{}

\begin{document}
\maketitle
\tableofcontents

\newpage

\chapter{Introduction}
\label{chap:intro}

LGA, with a unique syntax, is designed to make writing expressive manipulation of graphics and animation very easily. By targeting on Javacript, LGA can be viewed as an alternative to handle web graphical and animation task. 

The syntax of LGA is inspired by R\footnote{R-Project: \texttt{http://www.r-project.org/}}, Coffeescript\footnote{Coffeescript: \texttt{http://coffeescript.org/}} and Python\footnote{Python: \texttt{http://python.org/}}. It is designed to be minimalized and expressive, just focus on the target task. Compare to the target language Javascript, LGA provides a more succint, expressive and programmer-friendly syntax and core functions.

In this manual, we provide the comprehensive lexical and syntactic details of LGA, in order to easily generate the front-end part of the compiler. The following of this manual is structured as follows: Chapter \ref{chap:lex} consists of the lexical details of LGA; Chapter \ref{chap:grammar} consists of the syntactic details of LGA; Chapter \ref{chap:sample} gives a brief cample code.

\chapter{Lexical Conventions}
\label{chap:lex}


\section{Tokens}
\label{sec:token}

There are five classes of tokens: \texttt{identifiers, keywords, literals, operators} and \texttt{separators}. LGA use whitespaces as \texttt{separators}(similar to Python programming language). If the input stream has been separated into tokens up to a given character, the next token is the longest string of characters that could constitute a token.

\subsection{Comments}
\label{sec:comments}

The character \texttt{\#} introduce a comment and the next \texttt{\char`\\ n}(newline) terminates it. Comments do not occur within string or character literals.

\subsection{Identifiers}
\label{sec:identifiers}

Identifiers(names) must start with \texttt{\_} or any lowercase and uppercase letters. The rest of the string can contain the same characters plus number digits (\texttt{'0' - '9'}).

\subsection{Keywords}
\label{sec:keywords}

The following identifiers are reserved for the use as keywords, and may not be used otherwise. These include the superset of both JavaScript keywords and reserved words. 

\begin{table}[ch]
  \centering
  {\tt
  \begin{tabular}{l l l l}
    true       & false   & null    & this \\
    new        & delete  & typeof  & in  \\
    instanceof & return  & break   & continue \\
    if         & else    & switch  & for \\
    while   & do \\

  \end{tabular}
  }
  \caption{Keywords}
\end{table}
\subsection{Numeric Literals}
\label{sec:numeric_lit}

An integer constant consisting of a sequence of digits is taken to be octal if it begins with 0 (digit zero), decimal otherwise. A floating constant consists of an integer part, a decimal part, a fraction part, an \texttt{e} or \texttt{E}, an optionally signed integer exponent and an optional type suffix, one of \texttt{f, F, l}, or \texttt{L}. The integer and fraction parts both consist of a sequence of digits. Either the integer part, or the fraction part (not both) may be missing; either the decimal point or the \texttt{e} and the exponent (not both) may be missing.

\subsection{Character Literals}
\label{sec:char_lit}

A character literal is a single character surrounded by single quotes.

\subsection{String Literals}
\label{sec:str_lit}

A string constant is a sequence of characters surrounded by double quotes. Double quotation marks can be contained in strings surrounded by single quotation marks, and single quotation marks can be contained in strings surrounded by double quotation marks.


\section{Operators}
\label{sec:operators}

\subsection{Computational operators}
\begin{verbatim}
+ - * / %
\end{verbatim}


\subsection{Compound assignment operators}
\begin{verbatim}
-= += /= *= %= ||= &&= ?= <<= >>= &= ^= |=
\end{verbatim}

\subsection{Logical operators}
\begin{verbatim}
&& || & | ^ !
\end{verbatim}

\subsection{Comparison operators}
\begin{verbatim}
== != < > <= >=
\end{verbatim}

\chapter{Grammar}
\label{chap:grammar}

\section{Syntax Notation}
\label{sec:syn_notation}

In the syntax notation used in this manual, syntactic categories are indicated by \texttt{typewriter style} and forms as a capitalized word, as \texttt{Expression}. Literal words, tokens, and characters are in all upper letter words, as \texttt{TERMINATOR}.

\section{Program structure}
\label{sec:structure}

\subsection{Root}

The \texttt{Root} is the top-level node in the syntax tree. Since we parse bottom-up, all parsing must end here.
\begin{verbatim}
Root:
    NULL
    Body
    Block TERMINATOR
\end{verbatim}

\subsection{Body}

The \texttt{Body} node is any list of statements and expressions.
\begin{verbatim}
Body:
    Line
    Body TERMINATOR Line
    Body TERMINATOR
\end{verbatim}

\subsection{Line}

\begin{verbatim}
Line:
    Expression
    Statement
\end{verbatim}

\subsection{Statement}

\begin{verbatim}
Statement:
    Return
    Comment
    STATEMENT
\end{verbatim}

\subsection{Expression}

\begin{verbatim}
Expression:
    Value
    Invocation
    Code
    Operation
    Assign
    If
    While
    For
\end{verbatim}

\subsection{Code}

The Code node is the function literal. It's defined by an indented block of \textbf{Block} preceded by a function arrow, with an optional parameter list.

Function is a body of executable code which gets specific number of parameters then process statements inside the function body and return values if needed.  The following is some examples for functions:
\begin{verbatim}
sum = (x, y) -> x+y
getdouble = (x) -> sum x, x
givemefive = () -> 5
sayhey = (“hey”) -> “hey”
\end{verbatim}

In the first case, the function named sum, input parameters are x and y. For the second case which has name of getdouble, it has only one parameter which is x. Function body starts after the arrow operator. In the first case, statements in function body sum up two input parameters. In the second case, it calls sum, the function has been defined previously, and pass the input parameter x to sum. Of course, LGA supports functions without any input parameter like shown in the third case. In that case, no parameter will be defined and there will only be a pair of empty parentheses. Default parameters are also well covered in LGA just like the fourth case. 
A function could be called with its name followed by a pair of parentheses inside which contains parameters if defined. The second is an example of calling function sum and passing x as parameter.
At the end of function, the final value will be returned to the caller. For example, in the first case, the returned value will be (x+y). 
Parameters will be passed to functions by value. Any modification on the parameter inside a function will not put any influence on the original object/variable

\begin{verbatim}
Code:
    PARAM_START ParamList PARAM_END FuncGlyph Bock
    FuncGlyph Block
\end{verbatim}

\subsection{Value}
\texttt{Value} literal is the types of things that can be treated as values, which means they can be assigned to, invoked as functions, indexed into, etc.

\begin{verbatim}
Value:
    Assignable
    Literal
    Parenthetical
    This
\end{verbatim}

\subsection{Block}
\label{sec:block}

LGA use whitespaces as levels of identation. A \texttt{Block} is an indented block of expressions.
\begin{verbatim}
Block:
    INDENT OUTDENT
    INDENT Body OUTDENT
\end{verbatim}

\section{Identifiers,  Numerics and Literals}

\subsection{Identifier}
A literal identifier, which is a variable name or property.
\begin{verbatim}
Identifier:
     IDENTIFIER
\end{verbatim}

\subsection{AlphaNumeric}
\texttt{AlphaNumeric} is separated from the other \texttt{Literal} matchers because they can slaso serve as keys in object literals.
\begin{verbatim}
AlphaNumeric:
    NUMBER
    STRING
\end{verbatim}

\subsection{Literal}

All of immediate values. Generally these are fully campatible with our target language, which means can be printed
\begin{verbatim}
Literal:
    AlphaNumeric
    NULL
    BOOL
\end{verbatim}

\subsection{This}

\begin{verbatim}
This:
    THIS
    @
\end{verbatim}

\texttt{ThisProperty} is a reference to a property on this
\begin{verbatim}
ThisProperty:
    @ Identifier
\end{verbatim}

\section{Control Flow}

LGA supports common \texttt{If}, \texttt{While} and \texttt{For} loop as control flows.


\subsection{If}
Addition to regular if block, LGA supports \texttt{Post-If} style. For example \texttt{x = 2 if y > 3}. Code in block will be executed if evaluation result of expression is boolean true.
\begin{verbatim}
If:
    IfBlock
    IfBlock ELSE Block
    statement POST_IF Expression
    Expression POST_IF Expression
\end{verbatim}

\begin{verbatim}
IfBlock:
    IF Expression Block
    IfBlock ELSE IF Expression Block
\end{verbatim}

\subsection{While}
Similar to if block, \texttt{Post-While} style is support in LGA. For example \texttt{x = x * 2 while x < 100}. Code in block will be executed repeatedly if evaluation result of expression is boolean true.
\begin{verbatim}
While:
    WhileSource Block
    Statement WhileSource
    Expression WhileSource
\end{verbatim}

\begin{verbatim}
WhileSource:
    WHILE Expression
\end{verbatim}


\subsection{For}
For block iterate through \texttt{ForValue}. For example, in \texttt{for x in [1, 2 ,3, 4]}, x is repeated assigned as elements in the array. To iterate an \texttt{Object}, use two \texttt{ForValue}s separated by comma.
\begin{verbatim}
For:
    ForBody Block
\end{verbatim}

\begin{verbatim}
ForBody:
    ForStart ForSource
\end{verbatim}

\begin{verbatim}
ForStart:
    FOR ForVar
\end{verbatim}

\begin{verbatim}
ForVar:
    ForValue
    ForValue, ForValue
\end{verbatim}

\begin{verbatim}
ForValue:
    Identifier
    Array
    Object
\end{verbatim}

\begin{verbatim}
ForSource:
    FORIN Expression 
\end{verbatim}


\section{Others}

\subsection{Assign}
Assignment of a variable, property, or index to a value
\begin{verbatim}
Assign:
    Assignable = Expression
    Assignable = TERMINATOR Expression
    Assignable = INDENT Expression OUTDENT
\end{verbatim}


\subsection{Assignable}
This catagory consists of everything that can be assigned to.
\begin{verbatim}
Assignable:
    SimpleAssignable
    Array
    Object
\end{verbatim}

\begin{verbatim}
SimpleAssignable:
    Identifier
    ThisProperty
\end{verbatim}


\subsection{AssignObj}

Assignment when it happens within an object literal. The difference from the ordinary \texttt{Assign} is that these allow numbers and strings as keys. And we use \texttt{:} as assign operator here.
\begin{verbatim}
AssignObj:
    ObjAssignable
    ObjAssignable : Expression
    ObjAssignable : INDENT Expression OUTDENT
    Comment
\end{verbatim}

\subsection{ObjAssignment}
\begin{verbatim}
ObjAssignable:
    Identifiers
    AlphaNumeric
    ThisProperty
\end{verbatim}

\subsection{Array}

\begin{verbatim}
Array:
    [ ]
    [ ArgList ]
\end{verbatim}

\subsection{Object}

In LGA, an object literal is simply a list of assignments.
\begin{verbatim}
Object:
    { AssignList OptComma }
\end{verbatim}


\subsection{AssignList}

Assignment of properties within an object literal can be separated by comma, as in Javascript, or simply by newline
\begin{verbatim}
AssignList:
    NULL
    AssignObj
    AssignObj , AssignObj
    AssignList OptComma TERMINATOR AssignObj
    AssignList OptComma INDENT AssignList OptComma OUTDENT
\end{verbatim}


\subsection{OptComma}
An Optional, trailing comma.
\begin{verbatim}
OptComma:
    NULL
    ,
\end{verbatim}

\subsection{Return}

In LGA, functions will always return their final values even though when we don’t actually use any return statement or operator, as following: \texttt{sqr = (x) -> x*x}. The value of \texttt{x*x} will be returned to the caller as the final value of the function. As shown, flowing off the end of function then the final value will be returned.
Of course, return statement with which a function can return to the caller is also provided. Examples of using return statement are as following:
\begin{verbatim}
	return
	return ( expression )
\end{verbatim}

The first sample dose not return any value but just terminal the process. The second sample returns value of the expression to the caller. With a return statement, logical flow of a function could be easily controlled. When some specific cases are captured, function could be terminated with or without returning a value.

\begin{verbatim}
Return:
    RETURN Expression
    RETURN
\end{verbatim}


\subsection{Comment}
LGA only support inline comment, starting with a \texttt{\#} and terminates with a newline
\begin{verbatim}
Comment:
    COMMENT
\end{verbatim}

\subsection{Invocation}
Ordinary function invocation.

\begin{verbatim}
Invocation:
    Value Arguments
    Invocation Arguments
\end{verbatim}

\subsection{Arguments}

The list of arguments to a function call.
\begin{verbatim}
Arguments:
    CALL_START CALL_END
    CALL_START ArgList OptComma CALL_END
\end{verbatim}

\subsection{ArgList}

The \texttt{ArgList} is both the list of objects passed into a function call, as well as the contents of an array literal.
\begin{verbatim}
ArgList:
    Expression
    ArgList , Expression
    ArgList OptComma TERMINATOR Expression
    INDENT ArgList OptComma OUTDENT
    ArgList OptComma INDENT ArgList OptComma OUTDENT
\end{verbatim}

\subsection{FuncGlyph}

\begin{verbatim}
FuncGlyph:
    ->
\end{verbatim}

\subsection{ParamList}
The list of parameters that a function accepts can be of any length
\begin{verbatim}
ParamList:
    NULL
    Param
    ParamList , Param
    ParamList OptComma TERMINATOR Param
    ParamList OptComma INDENT ParamList OptComma OUTDENT
\end{verbatim}

\subsection{Param}
\begin{verbatim}
Param:
    ParamVar
    ParamVar = Expression
\end{verbatim}

\begin{verbatim}
ParamVar:
    Identifier
    Array
    Object
    ThisProperty
\end{verbatim}

\subsection{Index}
Indexing into an object or array using bracket notation.
\begin{verbatim}
Index:
    INDEX_START IndexValue INDEX_END
\end{verbatim}

\begin{verbatim}
IndexValue:
    Expression
\end{verbatim}


\subsection{Parenthetical}

\begin{verbatim}
Parenthetical:
    ( Body )
    ( INDENT Body OUTDENT )
\end{verbatim}



\chapter{Sample Code}
\label{chap:sample}

\begin{verbatim}
move_forward = (l) ->
                 if @pos && @vec
                       a = math.atan(@vec[0], @vec[1])
                       @pos[0] += math.cos(angle) * l
                       @pos[1] += math.sin(angle) * l
                 return

square = {
            run : circle
            pos : [2, 5]
            vec : [1, 1]
            delay : 5
         }

OBJ = [square]
OBJ.start()

\end{verbatim}


\end{document}


